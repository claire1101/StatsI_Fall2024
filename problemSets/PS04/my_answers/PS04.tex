### Problem set 4
getwd()
# remove objects
rm(list=ls())
# detach all libraries
detachAllPackages <- function() {
	basic.packages <- c("package:stats", "package:graphics", "package:grDevices", "package:utils", "package:datasets", "package:methods", "package:base")
	package.list <- search()[ifelse(unlist(gregexpr("package:", search()))==1, TRUE, FALSE)]
	package.list <- setdiff(package.list, basic.packages)
	if (length(package.list)>0)  for (package in package.list) detach(package,  character.only=TRUE)
}
detachAllPackages()

# load libraries
pkgTest <- function(pkg){
	new.pkg <- pkg[!(pkg %in% installed.packages()[,  "Package"])]
	if (length(new.pkg)) 
	install.packages(new.pkg,  dependencies = TRUE)
	sapply(pkg,  require,  character.only = TRUE)
}
# load packages
lapply(c("stargazer","vioplot","arm","broom","ggplot2","fastDummies"),  pkgTest)

# loading/running the programs as described in the assignment
install.packages("car")
library("car")
data(Prestige)
help(Prestige)
Prestige

## Question 1
# creating a new variable "professional" from "type" so that professionals = 1 and other (blue/white collar) = 0
Prestige$professional <- ifelse(Prestige$type == "prof", 1, 0)
# checking to see if it worked
print(Prestige)

# running a linear model with interaction
interact_reg <- lm(prestige ~ income * professional, data=Prestige)
summary(interact_reg)

# prediction equation
coefficients <- coef(interact_reg)
cat("prestige =", 
coefficients[1], "+",  
coefficients[2], "* income +",  
coefficients[3], "* professional +",  
coefficients[4], "* income * professional\n")  

## Writing the prediction equation
# ŷ = B0 + B1(income) + B2(professional) + B3(income x professional)
# Prestige = 21.14266 + 0.003170909 x income + 37.78128 x professional - 0.002325709 x (income x professional) 
# Interpreting the coefficient for income:
# The coefficient for income, which for this dataset is measured in (Canadian) dollars, measures the increase in prestige for every dollar earned. 
# Interpreting the coefficient for professional 
# The coefficient for professional represents the effect of being a professional (professional = 1) versus a blue or white collar worker (professional = 0) when #income is 0. 
# Calculating the change in ŷ (prestige score) when there’s a $1,000 increase in income and professional = 1
# Prestige = 21.14266 + 0.003170909 x income + 37.78128 - 0.002325709 x income 
# Solve for income = 1,000
# Prestige = 21.14266 + 0.003170909 x 1,000 + 37.78128 - 0.002325709 x 1,000
# Prestige = 59.76914
# Prestige score for a professional occupation will increase by approximately 59.8 when there is a $1,000 increase in income. 

# Calculating the effect of changing professions from non-professional to professional when income is $6,000:
# To do this, I need to find the predicted values of ŷ (prestige) for both groups and then find the difference between them. 
# For non-professional (professional = 0)
# Prestige = 21.14266 + 0.003170909 x income
# Prestige = 21.14266 + 0.003170909 x 6,000
# Prestige = 40.168144
# For professional (professional = 1)
# Prestige = 21.14266 + 0.003170909 x income + 37.78128 - 0.002325709 x income
# Prestige = 21.14266 + 0.003170909 x 6,000 + 37.78128 - 0.002325709 x 6,000
# Prestige = 63.99514
# Calculating the change in ŷ
# Δŷ = prestigeprofessional - prestigenon-professional
# Δŷ = 63.99514 - 40.168144
# ŷ = 23.826996 
# When income is $6,000, changing professions from non-professional to professional affects prestige, or ŷ, by 23.826996. 


## Question 2
# setting/defining the coefficients and se's
coef_signs <- 0.042
se_signs <- 0.016
coef_adjacent <- 0.042
se_adjacent <- 0.013
constant <- 0.302
r_squared <- 0.094
alpha <- 0.05
df <- 128 

#part a
t_signs <- coef_signs / se_signs
t_signs

#part b
t_adjacent <- coef_adjacent / se_adjacent 
t_adjacent

Determining if the yard signs have an effect on the vote share:
Coefficient for precincts with signs: B1 = 0.042
SE (standard error) for that coefficient: SE = 0.016
H0 = yard signs have no effect on vote share (B1 = 0)
Ha= yard signs have an effect on vote share (B1 ≠ 0)
Getting t-statistic
T = coefficient / SE → 0.042/0.016 = 2.625 = t
Getting degrees of freedom
df = N - k → 131 - 2 - 1 = 128
At a= 0.05, critical value for a 2-tailed test is approximately ± 1.98
Since |t| = 2.625 > 1.98, we reject the null hypothesis
Therefore, we can conclude that having the yard signs in a precinct affects the vote share. 
Determining whether being next to a precinct with the yard signs affects vote share:
Coefficient for precincts adjacent to signs: B2 = 0.042
SE (standard error) for that coefficient: SE = 0.013
H0 = being adjacent to yard signs has no effect on vote share (B2 = 0)
Ha= being adjacent to yard signs has an effect on vote share (B2 ≠ 0)
Getting t-statistic
T = coefficient / SE → 0.042/0.013 = 3.231 = t
Getting degrees of freedom
df = N - k → 131 - 2 - 1 = 128
At a= 0.05, critical value for a 2-tailed test is approximately ± 1.98
Since |t| = 3.231 > 1.98, we reject the null hypothesis
Therefore, we can conclude that being adjacent to a precinct with the yard signs affects the vote share. 
Interpreting the coefficient for the constant term:
The coefficient for the constant term, B0, is 0.302. This constant represents the expected vote share for Ken Cuccinelli in precincts without yard signs and that are not adjacent to yard signs. This baseline corresponds to 30.2% of the vote share.
Evaluating the model fit:
R2 is 0.094. This means that the model only explains 9.4% of the variation in vote share, suggesting that there are other unmodeled factors, such as demographic and/or voter preferences maybe, that have a more significant role than the presence or proximity to yard signs in determining the vote share in different precincts. In other words, while the findings above suggest that the yard signs are statistically significant to the vote share, they are not the most influential factor. 


