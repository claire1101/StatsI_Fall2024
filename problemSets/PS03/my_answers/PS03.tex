\textbf{Problem Set 3}
\textit{Question 1}
	1. # estimate the regression manually
	lm_by_hand <- function(data, predictors, outcome) {
		# ensure predictors is a vector of column names
		if (!is.character(predictors)) predictors <- as.character(predictors)
		
		# creating matrices
		X <- as.matrix(cbind(1, data[, predictors]))  # add a column of 1s for the intercept
		Y <- as.matrix(data[, outcome])
		
		# calculating betas (coefficients)
		betas <- solve(t(X) %*% X) %*% (t(X) %*% Y)
		rownames(betas) <- c("Intercept", predictors)
		
		# number of observations and parameters
		n <- nrow(X)
		k <- ncol(X)
		
		# estimating sigma^2 (variance of the residuals)
		residuals <- Y - X %*% betas
		sigma_squared <- sum(residuals^2) / (n - k)
		
		# covariance matrix for betas
		var_covar_mat <- sigma_squared * solve(t(X) %*% X)
		
		# SEs for coefficient estimates
		SEs <- sqrt(diag(var_covar_mat))
		
		# t-statistics and p-values
		t_stats <- betas / SEs
		p_values <- 2 * pt(abs(t_stats), df = n - k, lower.tail = FALSE)
		
		# return all results in a list
		return(list(
		coefficients = betas,
		standard_errors = SEs,
		t_statistics = t_stats,
		p_values = p_values,
		residuals = residuals,
		sigma_squared = sigma_squared,
		var_covar_matrix = var_covar_mat
		))
	}
	
	
	#trying this
	result1 <- lm_by_hand(data = incumbents, predictors = "difflog", outcome = "voteshare")
	result1
	# print results
	print(result1$coefficients)       # coefficients
	print(result1$standard_errors)    # SEs
	print(result1$t_statistics)       # t-statistics
	print(result1$p_values)           # p-values
	
	
	#trying the built-in lm function
	auto_results1 <- lm(voteshare ~ difflog, data= incumbents)
	summary(auto_results1)
	
	2. #making a scatterplot
	ggplot(incumbents, aes(x = difflog, y = voteshare)) +
	geom_point(size = 1) +  # adjust the size of the points
	geom_smooth(method = "lm", col = "blue") +
	labs(title = "Vote Share vs Difference in Log Spending", 
	x = "Difference in Log Spending", 
	y = "Vote Share")
	
% TODO: \usepackage{graphicx} required
\begin{figure}
	\centering
	\includegraphics[width=0.7\linewidth]{Users/claireott/Downloads/Q1scatterplot}
	\caption{}
	\label{fig:q1scatterplot}
\end{figure}

	3. #getting/saving residuals
	residuals1 <- resid(auto_results1)
	residuals1

	4. #writing the prediction equation
	predition = intercept + (slope x input value for difflog)
	\hat{y}= 0.579031 + 0.0461666 x difflog
	
\textit{Question 2}
	1. #running a regression where outcome variable is presvote and explanatory is difflog 
	#using the function from Q1
	result2 <- lm_by_hand(data = incumbents, predictors = "difflog", outcome = "presvote")
	result2
	# print results
	print(result2$coefficients)       # coefficients
	print(result2$standard_errors)    # SEs
	print(result2$t_statistics)       # t-statistics
	print(result2$p_values)           # p-values
	
	
	#trying the built-in lm function
	auto_results1 <- lm(voteshare ~ difflog, data= incumbents)
	summary(auto_results1)
	2. #making a scatterplot
	ggplot(incumbents, aes(x = difflog, y = presvote)) +
	geom_point(size = 1) +  # adjust the size of the points
	geom_smooth(method = "lm", col = "blue") +
	labs(title = "Vote Share vs Difference in Log Spending", 
	x = "Difference in Log Spending", 
	y = "Vote Share")
	
% TODO: \usepackage{graphicx} required
\begin{figure}
	\centering
	\includegraphics[width=0.7\linewidth]{Users/claireott/Downloads/Q2scatterplot}
	\caption{}
	\label{fig:q2scatterplot}
\end{figure}
	3. #getting/saving residuals
	residuals2 <- resid(auto_results2)
	residuals2
	4. #writing the prediction equation
	predition = intercept + (slope x input value for difflog)
	\hat{y}= 0.507583 + 0.023837 x difflog
	
\textit{Question 3}
	1. #running a regression where outcome variable is voteshare and explanatory is presvote 
	#using the function from Q1
	result3 <- lm_by_hand(data = incumbents, predictors = "presvote", outcome = "voteshare")
	result3
	# print results
	print(result3$coefficients)       # coefficients
	print(result3$standard_errors)    # SEs
	print(result3$t_statistics)       # t-statistics
	print(result3$p_values)           # p-values
	
	#trying the built-in lm function
	auto_results3 <- lm(voteshare ~ presvote, data= incumbents)
	summary(auto_results3)
	
	2. #making a scatterplot
	ggplot(incumbents, aes(x = presvote, y = voteshare)) +
	geom_point(size = 1) +  # adjust the size of the points
	geom_smooth(method = "lm", col = "blue") +
	labs(title = "Vote Share vs Electoral Success", 
	x = "Electoral Success", 
	y = "Vote Share")
	
% TODO: \usepackage{graphicx} required
\begin{figure}
	\centering
	\includegraphics[width=0.7\linewidth]{Users/claireott/Downloads/Q3scatterplot}
	\caption{}
	\label{fig:q3scatterplot}
\end{figure}
	3. #getting/saving residuals
	residuals3 <- resid(auto_results3)
	residuals3
	4. #writing the prediction equation
	predition = intercept + (slope x input value for presvote)
	\hat{y}= 0.441330 + 0.388018 x presvote
	
\textit{Question 4}
	1. #running a regression where outcome variable is Q1 residuals and explanatory is Q2 residuals 
	## with the built-in lm function
	auto_results4 <- lm(residuals1 ~ residuals2)
	summary(auto_results4)
	2. #making a scatterplot
	ggplot(incumbents, aes(x = residuals2, y = residuals1)) +
	geom_point(size = 1) +  # adjust the size of the points
	geom_smooth(method = "lm", col = "blue") +
	labs(title = "Q1 Residuals vs Q2 Residuals", 
	x = "Q2 residuals", 
	y = "Q1 residuals")
% TODO: \usepackage{graphicx} required
\begin{figure}
	\centering
	\includegraphics[width=0.7\linewidth]{Users/claireott/Downloads/Q4scatterplot}
	\caption{}
	\label{fig:q4scatterplot}
\end{figure}
	3. #Writing the predicution equation
	predition = intercept + (slope x input value for residuals2)
	\hat{y}= 3.876e-17 + 2.569e-01 x residuals2

\textit{Question 5}
	1. #running a regression where outcome variable is voteshare and explanatory are difflog and presvote 
	#using the function from Q1
	result5 <- lm_by_hand(data = incumbents, predictors = c("difflog", "presvote"), outcome = "voteshare")
	result5
	# print results
	print(result5$coefficients)       # coefficients
	print(result5$standard_errors)    # SEs
	print(result5$t_statistics)       # t-statistics
	print(result5$p_values)           # p-values
	
	#trying the built-in lm function
	auto_results5 <- lm(voteshare ~ difflog + presvote, data = incumbents)
	summary(auto_results5)
	
	2. # write the prediction equation
	cat("Prediction equation: voteshare =", coef(auto_results5)[1], "+", coef(auto_results5)[2], "* difflog +", coef(auto_results5)[3], "* presvote")
	voteshare = 0.4486442 + 0.03554309 * difflog + 0.256877 * presvote
	
	3. #comparing with results from Q4 to see what is identical
	# compare the coefficients
	identical(auto_results4$coefficients, auto_results5$coefficients)  # TRUE if identical
	# compare the residuals
	identical(auto_results4$residuals, auto_results5$residuals)  # TRUE if identical
	# compare the variance-covariance matrices
	identical(auto_results4$var_covar_matrix, auto_results5$var_covar_matrix)  # TRUE if identical
	# compare sigma_squared (estimated residual variance)
	identical(auto_results4$sigma_squared, auto_results5$sigma_squared)  # TRUE if identical
	
	
## I don't think that I set this up properly in Latex, the program keeps crashing on my laptop and I wasn't able to download it onto any of the 
## tcd computer lab computers to try and complete the assignment there so the PDF uploaded is probably better to look at for my answers. 
	
	
	
	
	
	
	
	